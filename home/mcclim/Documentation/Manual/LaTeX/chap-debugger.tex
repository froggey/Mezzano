
\chapter{Debugger}

The debugger is used for interactively inspecting stack frame when the
unhandled conditions are signalled. Given high enough debug settings
it lets you inspecting frame local variables, evaluating code in it,
examining backtrace and choosing available restarts.

\section{Debugger usage}

To get up and running quickly with Debugger:

\itemize
\item
 With Quicklisp loaded, invoke in repl:

  \texttt{(ql:quickload 'clim-debugger)}
\item
Run simple test condition:

  \texttt{(clim-debugger:with-debugger (error "test"))}
\end itemize

Debugger is highly inspired by Slime and uses Swank to gain
portability across implementations. Module is still under development
and some details may change in the future.

Selecting frame with a pointer switches its details and marks it
active. Each locale value may be inspected by selection with mouse
pointer. Active frame is distinguished from others with red
color. \texttt{Eval in frame} command evaluates expression in the
active frame.

\section{Keyboard shortcuts}

Warning: these key accelerators may change in the future.


\begin{itemize}
\item [M-m] Show more frames
\item [M-p] Mark previous frame active
\item [M-n] Mark next frame active
\item [M-e] Eval in active frame
\item [TAB] Toggle active frame details
\item [M-[0-9]] Invoke nth restart
\item [M-q] Quit debugger
\end{itemize}

\section{API}

\Defun {debugger} {condition me-or-my-encapsulation}

Starts debugger with supplied condition. Second argument should be
supplied by an underlying implementation allowing to encapsulate or
supply different debugger for recursive debugger calls.

\Defmacro {with-debugger} {&bodybody }

Wraps the code in \texttt{body} to invoke clim debugger when condition
is signalled (binds \texttt{*debugger-hook*} to \texttt{#'debugger}).

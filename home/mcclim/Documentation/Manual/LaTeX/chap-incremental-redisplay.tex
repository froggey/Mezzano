\chapter{Incremental redisplay}

CLIM applications are most often structured around the \emph{command
  loop}.  The various steps that such an application follow are:

\begin{itemize}
\item A \emph{command} is acquired because the user, either typed the
  name of the command to an interactive prompt, selected a menu item
  representing a command, or clicked on an active object that
  translates to a command.
\item The \emph{arguments} to that command are acquired.  As with the
  acquisition of the command itself, various gestures can be used to
  supply the arguments.
\item The command is \emph{executed} with the acquired arguments.
  Typically, the command modifies some part of the \emph{model}%
\footnote{Some authors use the term \emph{business logic} instead of
  \emph{model}.  Both words refer to the representation of the
  intrinsic purpose of the application, as opposed to superficial
  characteristics such as how objects are physically presented to the
  user.} contained in one or more slots in the application frame.
\item The \emph{redisplay functions} associated with the visible panes
  of the application are executed.  Typically, the redisplay function
  erases all the output and traverses the entire model in order to
  produce a new version of that output.  Since output exists in the
  form of \emph{output records}, this process involves deleting the
  existing output records and computing an entirely new set of them.
\end{itemize}

This way of structuring an application is very simple.  The resulting
code is very easy to understand, and the relationship between the code
of a redisplay function and the output it produces is usually obvious.
The concept of output records storing the output in the application
pane is completely hidden, and instead output is produced using
textual or graphic drawing functions, or more often produced
indirectly through the use of \texttt{present} or
\texttt{with-output-as-presentation}.

However, if the model contains a large number of objects, then this
simple way of structuring an application may penalize performance.  In
most libraries for creating graphic user interfaces, the application
programmer must then rewrite the code for manipulating the model, and
especially for incrementally altering the output according to the
modification of the model resulting from the execution of a command.

In CLIM, a different mechanism is provided called \emph{incremental
  redisplay}.  This mechanism allows the user to preserve the simple
logic of the display function with only minor modifications while
still being able to benefit in terms of performance.
